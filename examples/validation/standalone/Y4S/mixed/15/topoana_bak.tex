\documentclass[landscape]{article}
\usepackage{/home/belle2/zhouxy/workarea/tools/topoana/topoana-02-05-04/share/geometry}
\usepackage[colorlinks,linkcolor=blue]{hyperref}
\usepackage{longtable}
\usepackage{/home/belle2/zhouxy/workarea/tools/topoana/topoana-02-05-04/share/makecell}
\usepackage{color}
\usepackage{amssymb} % The package is used for the \dashrightarrow
\newcommand{\tablecaption}[1]{\caption{#1} \\}
\usepackage{caption}
\captionsetup{font=normalsize}
\newcommand{\tableheader}[1]
{
  \hline
  #1
  \hline
  \endfirsthead

  \hline
  #1
  \hline
  \endhead

  \endfoot

  \endlastfoot
}
% The following command \tableheaderP is developed as a substitute for \tableheader, in cases that hlines are needed at the bottom of the pages (except for the last one) of the tables when \EOL is defined and used as \\ with no \hline followed. In the command \tableheaderP, P is the initial of PRIME.
\newcommand{\tableheaderP}[1]
{
  \hline
  #1
  \hline
  \endfirsthead

  \hline
  #1
  \hline
  \endhead

  \hline % This is the only difference between \tableheader and \tableheaderP.
  \endfoot

  \endlastfoot
}
\setcellgapes[t]{2pt}
\makegapedcells
\newcounter{rownumbers}
\newcommand\rn{\stepcounter{rownumbers}\arabic{rownumbers}}
% \newcommand{\EOL}{\\ \hline} % Use this definition to retain hlines in the body part.
\newcommand{\EOL}{\\} % Use this definition to remove hlines in the body part.
% The following command \EOLP is developed as a supplementary to \EOL, in cases that hlines are needed in some tables (nLineMean>2, the average number of lines occupied by one decay object is larger than two) when \EOL is only defined as \\ with no \hline followed for other tables. In the command \EOLP, P is the initial of PRIME.
\newcommand{\EOLP}{\\ \hline} % Use this definition to retain hlines in the body part.
% \newcommand{\EOLP}{\\} % Use this definition to remove hlines in the body part.
\newcommand{\topoTags}[1]{#1} % Use this definition to retain topology tags.
% \newcommand{\topoTags}[1]{} % Use this definition to remove topology tags.
\begin{document}
\title{Topology Analysis \footnote{\normalsize{Xingyu Zhou, Beihang University, zhouxy@buaa.edu.cn}} \footnote{\normalsize{This package is implemented referring to a program called {\sc Topo}, which is first developed by Prof. Shuxian Du from Zhengzhou University and later extended and maintained by Prof. Gang Li from Institute of High Energy Physics, Chinese Academy of Sciences. The {\sc Topo} program has been widely used by colleagues in BESIII collaboration. Several years ago, when I was a Ph.D. student working on the BESIII experiment, I learned the idea of topology analysis and a lot of programming techniques from the {\sc Topo} program. So, I really appreciate the original works of Prof. Du and Prof. Li very much. To meet my own needs and to practice developing analysis tools with C++, ROOT, and LaTeX, I wrote the package from scratch. At that time, the package functioned well but was relatively simple. At the end of 2017, my co-supervisor, Prof. Chengping Shen reminded me that it could be a useful tool for the Belle II experiment as well. So, I revised and extended it, making it more well-rounded and suitable for the Belle II experiment. Here, I would like to thank Prof. Du and Prof. Li for their original works, and Prof. Shen for his suggestion, guidance, support, and encouragement.}} \footnote{\normalsize{Besides, I would like to thank all of the people who have helped me in the development of the program. I am particularly grateful to Prof. Xingtao Huang for his comments on the principles and styles of the program, to Remco de Boer for his suggestions on the tex output and the use of GitHub, and to Xi Chen for his discussions on the core algorithms. I am especially indebted to Prof. Xiqing Hao, Longke Li, Xiaoping Qin, Ilya Komarov, Yubo Li, Guanda Gong, Suxian Li, Junhao Yin, Prof. Xiaolong Wang, Yeqi Chen, and Hannah Wakeling for their advice in extending and perfecting the program. Also, I thank Xi'an Xiong, Runqiu Ma, Wencheng Yan, Sen Jia, Lu Cao, Dong Liu, Hongpeng Wang, Jiawei Zhang, Hongrong Qi, Jiajun Liu, Maoqiang Jing, Yi Zhang, Wei Shan, and Yadi Wang for their efforts in helping me test the program.}} \\ \vspace{1cm} \Large{(v2.5.4)}}
\maketitle

\clearpage

\newgeometry{left=2.5cm,right=2.5cm,top=2.5cm,bottom=2.5cm}

\listoftables

\newgeometry{left=0.0cm,right=0.0cm,top=2.5cm,bottom=2.5cm}

\clearpage

\small
\centering
\setcounter{rownumbers}{0}
\begin{longtable}{ccccccc}
\tablecaption{Decay trees and their respective final states.}
\tableheaderP{rowNo & \thead{decay tree} & \thead{decay final state} & \topoTags{iDcyTr & }nEtr & nCEtr \\}

% \rn = 1
\rn & \makecell[c]{ $ 
\Upsilon(4S) \rightarrow B^{0} \bar{B}^{0} ,
B^{0} \rightarrow \nu_{e} \bar{\nu}_{e} K^{*} ,
\bar{B}^{0} \rightarrow \mu^{+} \mu^{-} ,
K^{*} \rightarrow \pi^{-} K^{+} 
$ } & $
\nu_{e} \bar{\nu}_{e} \mu^{+} \mu^{-} \pi^{-} K^{+} 
$ & \topoTags{0 & }1 & 1 \\ \hline

\end{longtable}
\end{document}
